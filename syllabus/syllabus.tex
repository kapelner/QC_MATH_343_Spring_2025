\documentclass[12pt]{article}

\usepackage[margin=1in]{geometry}
%packages
%\usepackage{latexsym}
\usepackage{graphicx}
\usepackage{wrapfig}
\usepackage{color}
\usepackage{amsmath}
\usepackage{dsfont}
\usepackage{placeins}
\usepackage{amssymb}
\usepackage{skull}
\usepackage{enumerate}
\usepackage{soul}
\usepackage{alphalph}
\usepackage{hyperref}
\usepackage{enumerate}
\usepackage{listings}
\usepackage{multicol}
\usepackage{etoolbox}
%\usepackage{fancyhdr}

%\fancyhf{} % clear all header and footers
%\renewcommand{\headrulewidth}{0pt} % remove the header rule
%\fancyfoot[LE, LO]{\thepage}


%\usepackage{pstricks,pst-node,pst-tree}

%\usepackage{algpseudocode}
%\usepackage{amsthm}
%\usepackage{hyperref}
%\usepackage{mathrsfs}
%\usepackage{amsfonts}
%\usepackage{bbding}
%\usepackage{listings}
%\usepackage{appendix}
\usepackage[margin=1in]{geometry}
%\geometry{papersize={8.5in,11in},total={6.5in,9in}}
\usepackage{cancel}
%\usepackage{algorithmic, algorithm}

\definecolor{dkgreen}{rgb}{0,0.6,0}
\definecolor{gray}{rgb}{0.5,0.5,0.5}
\definecolor{mauve}{rgb}{0.58,0,0.82}
\lstset{ %
  language=R,                     % the language of the code
  basicstyle=\footnotesize,       % the size of the fonts that are used for the code
  numbers=left,                   % where to put the line-numbers
  numberstyle=\tiny\color{gray},  % the style that is used for the line-numbers
  stepnumber=1,                   % the step between two line-numbers. If it's 1, each line
                                  % will be numbered
  numbersep=5pt,                  % how far the line-numbers are from the code
  backgroundcolor=\color{white},  % choose the background color. You must add \usepackage{color}
  showspaces=false,               % show spaces adding particular underscores
  showstringspaces=false,         % underline spaces within strings
  showtabs=false,                 % show tabs within strings adding particular underscores
  frame=single,                   % adds a frame around the code
  rulecolor=\color{black},        % if not set, the frame-color may be changed on line-breaks within not-black text (e.g. commens (green here))
  tabsize=2,                      % sets default tabsize to 2 spaces
  captionpos=b,                   % sets the caption-position to bottom
  breaklines=true,                % sets automatic line breaking
  breakatwhitespace=false,        % sets if automatic breaks should only happen at whitespace
  title=\lstname,                 % show the filename of files included with \lstinputlisting;
                                  % also try caption instead of title
  keywordstyle=\color{black},      % keyword style
  commentstyle=\color{dkgreen},   % comment style
  stringstyle=\color{mauve},      % string literal style
  escapeinside={\%*}{*)},         % if you want to add a comment within your code
  morekeywords={*,...}            % if you want to add more keywords to the set
}

\newcommand{\qu}[1]{``#1''}
\newcommand{\spc}[1]{\\ \vspace{#1cm}}

\newcounter{probnum}
\setcounter{probnum}{1}

%create definition to allow local margin changes
\def\changemargin#1#2{\list{}{\rightmargin#2\leftmargin#1}\item[]}
\let\endchangemargin=\endlist 

%allow equations to span multiple pages
\allowdisplaybreaks

%define colors and color typesetting conveniences
\definecolor{gray}{rgb}{0.5,0.5,0.5}
\definecolor{black}{rgb}{0,0,0}
\definecolor{white}{rgb}{1,1,1}
\definecolor{blue}{rgb}{0.5,0.5,1}
\newcommand{\inblue}[1]{\color{blue}#1 \color{black}}
\definecolor{green}{rgb}{0.133,0.545,0.133}
\newcommand{\ingreen}[1]{\color{green}#1 \color{black}}
\definecolor{yellow}{rgb}{1,0.549,0}
\newcommand{\inyellow}[1]{\color{yellow}#1 \color{black}}
\definecolor{red}{rgb}{1,0.133,0.133}
\newcommand{\inred}[1]{\color{red}#1 \color{black}}
\definecolor{purple}{rgb}{0.58,0,0.827}
\newcommand{\inpurple}[1]{\color{purple}#1 \color{black}}
\definecolor{gray}{rgb}{0.5,0.5,0.5}
\newcommand{\ingray}[1]{\color{gray}#1 \color{black}}
\definecolor{backgcode}{rgb}{0.97,0.97,0.8}
\definecolor{Brown}{cmyk}{0,0.81,1,0.60}
\definecolor{OliveGreen}{cmyk}{0.64,0,0.95,0.40}
\definecolor{CadetBlue}{cmyk}{0.62,0.57,0.23,0}

%define new math operators
\DeclareMathOperator*{\argmax}{arg\,max~}
\DeclareMathOperator*{\argmin}{arg\,min~}
\DeclareMathOperator*{\argsup}{arg\,sup~}
\DeclareMathOperator*{\arginf}{arg\,inf~}
\DeclareMathOperator*{\convolution}{\text{\Huge{$\ast$}}}
\newcommand{\infconv}[2]{\convolution^\infty_{#1 = 1} #2}
%true functions

%%%% GENERAL SHORTCUTS

\makeatletter
\newalphalph{\alphmult}[mult]{\@alph}{26}
\renewcommand{\labelenumi}{(\alphmult{\value{enumi}})}
\renewcommand{\theenumi}{\AlphAlph{\value{enumi}}}
\makeatother
%shortcuts for pure typesetting conveniences
\newcommand{\bv}[1]{\boldsymbol{#1}}

%shortcuts for compound constants
\newcommand{\BetaDistrConst}{\dfrac{\Gamma(\alpha + \beta)}{\Gamma(\alpha)\Gamma(\beta)}}
\newcommand{\NormDistrConst}{\dfrac{1}{\sqrt{2\pi\sigma^2}}}

%shortcuts for conventional symbols
\newcommand{\tsq}{\tau^2}
\newcommand{\tsqh}{\hat{\tau}^2}
\newcommand{\sigsq}{\sigma^2}
\newcommand{\sigsqsq}{\parens{\sigma^2}^2}
\newcommand{\sigsqovern}{\dfrac{\sigsq}{n}}
\newcommand{\tausq}{\tau^2}
\newcommand{\tausqalpha}{\tau^2_\alpha}
\newcommand{\tausqbeta}{\tau^2_\beta}
\newcommand{\tausqsigma}{\tau^2_\sigma}
\newcommand{\betasq}{\beta^2}
\newcommand{\sigsqvec}{\bv{\sigma}^2}
\newcommand{\sigsqhat}{\hat{\sigma}^2}
\newcommand{\sigsqhatmlebayes}{\sigsqhat_{\text{Bayes, MLE}}}
\newcommand{\sigsqhatmle}[1]{\sigsqhat_{#1, \text{MLE}}}
\newcommand{\bSigma}{\bv{\Sigma}}
\newcommand{\bSigmainv}{\bSigma^{-1}}
\newcommand{\thetavec}{\bv{\theta}}
\newcommand{\thetahat}{\hat{\theta}}
\newcommand{\thetahatmm}{\hat{\theta}^{\mathrm{MM}}}
\newcommand{\thetahathatmm}{\thetahathat^{\mathrm{MM}}}
\newcommand{\thetahathatmle}{\thetahathat^{\mathrm{MLE}}}
\newcommand{\thetahatmle}{\hat{\theta}^{\mathrm{MLE}}}
\newcommand{\thetavechatmle}{\hat{\thetavec}^{\mathrm{MLE}}}
\newcommand{\muhat}{\hat{\mu}}
\newcommand{\muhathat}{\doublehat{\mu}}
\newcommand{\musq}{\mu^2}
\newcommand{\muvec}{\bv{\mu}}
\newcommand{\muhatmle}{\muhat_{\text{MLE}}}
\newcommand{\lambdahat}{\hat{\lambda}}
\newcommand{\lambdahatmle}{\lambdahat_{\text{MLE}}}
\newcommand{\thetahatmap}{\hat{\theta}_{\mathrm{MAP}}}
\newcommand{\thetahatmmae}{\hat{\theta}_{\mathrm{MMAE}}}
\newcommand{\thetahatmmse}{\hat{\theta}_{\mathrm{MMSE}}}
\newcommand{\etavec}{\bv{\eta}}
\newcommand{\alphavec}{\bv{\alpha}}
\newcommand{\minimaxdec}{\delta^*_{\mathrm{mm}}}
\newcommand{\ybar}{\bar{y}}
\newcommand{\xbar}{\bar{x}}
\newcommand{\Xbar}{\bar{X}}
\newcommand{\iid}{~{\buildrel iid \over \sim}~}
\newcommand{\inddist}{~{\buildrel ind \over \sim}~}
\newcommand{\approxdist}{~{\buildrel \bv{\cdot} \over \sim}~}
\newcommand{\equalsindist}{~{\buildrel d \over =}~}
\newcommand{\loglik}[1]{\ell\parens{#1}}
\newcommand{\thetahatkminone}{\thetahat^{(k-1)}}
\newcommand{\thetahatkplusone}{\thetahat^{(k+1)}}
\newcommand{\thetahatk}{\thetahat^{(k)}}
\newcommand{\half}{\frac{1}{2}}
\newcommand{\third}{\frac{1}{3}}
\newcommand{\twothirds}{\frac{2}{3}}
\newcommand{\fourth}{\frac{1}{4}}
\newcommand{\fifth}{\frac{1}{5}}
\newcommand{\sixth}{\frac{1}{6}}

%shortcuts for vector and matrix notation
\newcommand{\A}{\bv{A}}
\newcommand{\At}{\A^T}
\newcommand{\Ainv}{\inverse{\A}}
\newcommand{\B}{\bv{B}}
\renewcommand{\b}{\bv{b}}
\renewcommand{\H}{\bv{H}}
\newcommand{\K}{\bv{K}}
\newcommand{\Kt}{\K^T}
\newcommand{\Kinv}{\inverse{K}}
\newcommand{\Kinvt}{(\Kinv)^T}
\newcommand{\M}{\bv{M}}
\newcommand{\Bt}{\B^T}
\newcommand{\Q}{\bv{Q}}
\newcommand{\Qt}{\Q^T}
\newcommand{\R}{\bv{R}}
\newcommand{\Rt}{\R^T}
\newcommand{\Z}{\bv{Z}}
\newcommand{\X}{\bv{X}}
\newcommand{\Xsub}{\X_{\text{(sub)}}}
\newcommand{\Xsubadj}{\X_{\text{(sub,adj)}}}
\newcommand{\I}{\bv{I}}
\newcommand{\Y}{\bv{Y}}
\newcommand{\sigsqI}{\sigsq\I}
\renewcommand{\P}{\bv{P}}
\newcommand{\Psub}{\P_{\text{(sub)}}}
\newcommand{\Pt}{\P^T}
\newcommand{\Pii}{P_{ii}}
\newcommand{\Pij}{P_{ij}}
\newcommand{\IminP}{(\I-\P)}
\newcommand{\Xt}{\bv{X}^T}
\newcommand{\XtX}{\Xt\X}
\newcommand{\XtXinv}{\parens{\Xt\X}^{-1}}
\newcommand{\XtXinvXt}{\XtXinv\Xt}
\newcommand{\XXtXinvXt}{\X\XtXinvXt}
\newcommand{\x}{\bv{x}}
\newcommand{\w}{\bv{w}}
\newcommand{\q}{\bv{q}}
\newcommand{\zerovec}{\bv{0}}
\newcommand{\onevec}{\bv{1}}
\newcommand{\oneton}{1, \ldots, n}
\newcommand{\yoneton}{y_1, \ldots, y_n}
\newcommand{\yonetonorder}{y_{(1)}, \ldots, y_{(n)}}
\newcommand{\Yoneton}{Y_1, \ldots, Y_n}
\newcommand{\iinoneton}{i \in \braces{\oneton}}
\newcommand{\onetom}{1, \ldots, m}
\newcommand{\jinonetom}{j \in \braces{\onetom}}
\newcommand{\xoneton}{x_1, \ldots, x_n}
\newcommand{\Xoneton}{X_1, \ldots, X_n}
\newcommand{\xt}{\x^T}
\newcommand{\y}{\bv{y}}
\newcommand{\yt}{\y^T}
\renewcommand{\c}{\bv{c}}
\newcommand{\ct}{\c^T}
\newcommand{\tstar}{\bv{t}^*}
\renewcommand{\u}{\bv{u}}
\renewcommand{\v}{\bv{v}}
\renewcommand{\a}{\bv{a}}
\newcommand{\s}{\bv{s}}
\newcommand{\yadj}{\y_{\text{(adj)}}}
\newcommand{\xjadj}{\x_{j\text{(adj)}}}
\newcommand{\xjadjM}{\x_{j \perp M}}
\newcommand{\yhat}{\hat{\y}}
\newcommand{\yhatsub}{\yhat_{\text{(sub)}}}
\newcommand{\yhatstar}{\yhat^*}
\newcommand{\yhatstarnew}{\yhatstar_{\text{new}}}
\newcommand{\z}{\bv{z}}
\newcommand{\zt}{\z^T}
\newcommand{\bb}{\bv{b}}
\newcommand{\bbt}{\bb^T}
\newcommand{\bbeta}{\bv{\beta}}
\newcommand{\beps}{\bv{\epsilon}}
\newcommand{\bepst}{\beps^T}
\newcommand{\e}{\bv{e}}
\newcommand{\Mofy}{\M(\y)}
\newcommand{\KofAlpha}{K(\alpha)}
\newcommand{\ellset}{\mathcal{L}}
\newcommand{\oneminalph}{1-\alpha}
\newcommand{\SSE}{\text{SSE}}
\newcommand{\SSEsub}{\text{SSE}_{\text{(sub)}}}
\newcommand{\MSE}{\text{MSE}}
\newcommand{\RMSE}{\text{RMSE}}
\newcommand{\SSR}{\text{SSR}}
\newcommand{\SST}{\text{SST}}
\newcommand{\JSest}{\delta_{\text{JS}}(\x)}
\newcommand{\Bayesest}{\delta_{\text{Bayes}}(\x)}
\newcommand{\EmpBayesest}{\delta_{\text{EmpBayes}}(\x)}
\newcommand{\BLUPest}{\delta_{\text{BLUP}}}
\newcommand{\MLEest}[1]{\hat{#1}_{\text{MLE}}}

%shortcuts for Linear Algebra stuff (i.e. vectors and matrices)
\newcommand{\twovec}[2]{\bracks{\begin{array}{c} #1 \\ #2 \end{array}}}
\newcommand{\threevec}[3]{\bracks{\begin{array}{c} #1 \\ #2 \\ #3 \end{array}}}
\newcommand{\fivevec}[5]{\bracks{\begin{array}{c} #1 \\ #2 \\ #3 \\ #4 \\ #5 \end{array}}}
\newcommand{\twobytwomat}[4]{\bracks{\begin{array}{cc} #1 & #2 \\ #3 & #4 \end{array}}}
\newcommand{\threebytwomat}[6]{\bracks{\begin{array}{cc} #1 & #2 \\ #3 & #4 \\ #5 & #6 \end{array}}}

%shortcuts for conventional compound symbols
\newcommand{\thetainthetas}{\theta \in \Theta}
\newcommand{\reals}{\mathbb{R}}
\newcommand{\complexes}{\mathbb{C}}
\newcommand{\rationals}{\mathbb{Q}}
\newcommand{\integers}{\mathbb{Z}}
\newcommand{\naturals}{\mathbb{N}}
\newcommand{\forallninN}{~~\forall n \in \naturals}
\newcommand{\forallxinN}[1]{~~\forall #1 \in \reals}
\newcommand{\matrixdims}[2]{\in \reals^{\,#1 \times #2}}
\newcommand{\inRn}[1]{\in \reals^{\,#1}}
\newcommand{\mathimplies}{\quad\Rightarrow\quad}
\newcommand{\mathlogicequiv}{\quad\Leftrightarrow\quad}
\newcommand{\eqncomment}[1]{\quad \text{(#1)}}
\newcommand{\limitn}{\lim_{n \rightarrow \infty}}
\newcommand{\limitN}{\lim_{N \rightarrow \infty}}
\newcommand{\limitd}{\lim_{d \rightarrow \infty}}
\newcommand{\limitt}{\lim_{t \rightarrow \infty}}
\newcommand{\limitsupn}{\limsup_{n \rightarrow \infty}~}
\newcommand{\limitinfn}{\liminf_{n \rightarrow \infty}~}
\newcommand{\limitk}{\lim_{k \rightarrow \infty}}
\newcommand{\limsupn}{\limsup_{n \rightarrow \infty}}
\newcommand{\limsupk}{\limsup_{k \rightarrow \infty}}
\newcommand{\floor}[1]{\left\lfloor #1 \right\rfloor}
\newcommand{\ceil}[1]{\left\lceil #1 \right\rceil}

%shortcuts for environments
\newcommand{\beqn}{\vspace{-0.25cm}\begin{eqnarray*}}
\newcommand{\eeqn}{\end{eqnarray*}}
\newcommand{\bneqn}{\vspace{-0.25cm}\begin{eqnarray}}
\newcommand{\eneqn}{\end{eqnarray}}
\newcommand{\benum}{\begin{itemize}}
\newcommand{\eenum}{\end{itemize}}

%shortcuts for mini environments
\newcommand{\parens}[1]{\left(#1\right)}
\newcommand{\squared}[1]{\parens{#1}^2}
\newcommand{\tothepow}[2]{\parens{#1}^{#2}}
\newcommand{\prob}[1]{\mathbb{P}\parens{#1}}
\newcommand{\littleo}[1]{o\parens{#1}}
\newcommand{\bigo}[1]{O\parens{#1}}
\newcommand{\Lp}[1]{\mathbb{L}^{#1}}
\renewcommand{\arcsin}[1]{\text{arcsin}\parens{#1}}
\newcommand{\prodonen}[2]{\bracks{\prod_{#1=1}^n #2}}
\newcommand{\mysum}[4]{\sum_{#1=#2}^{#3} #4}
\newcommand{\sumonen}[2]{\sum_{#1=1}^n #2}
\newcommand{\infsum}[2]{\sum_{#1=1}^\infty #2}
\newcommand{\infprod}[2]{\prod_{#1=1}^\infty #2}
\newcommand{\infunion}[2]{\bigcup_{#1=1}^\infty #2}
\newcommand{\infinter}[2]{\bigcap_{#1=1}^\infty #2}
\newcommand{\infintegral}[2]{\int^\infty_{-\infty} #2 ~\text{d}#1}
\newcommand{\supthetas}[1]{\sup_{\thetainthetas}\braces{#1}}
\newcommand{\bracks}[1]{\left[#1\right]}
\newcommand{\braces}[1]{\left\{#1\right\}}
\newcommand{\angbraces}[1]{\left<#1\right>}
\newcommand{\set}[1]{\left\{#1\right\}}
\newcommand{\abss}[1]{\left|#1\right|}
\newcommand{\norm}[1]{\left|\left|#1\right|\right|}
\newcommand{\normsq}[1]{\norm{#1}^2}
\newcommand{\inverse}[1]{\parens{#1}^{-1}}
\newcommand{\rowof}[2]{\parens{#1}_{#2\cdot}}

%shortcuts for functionals
\newcommand{\realcomp}[1]{\text{Re}\bracks{#1}}
\newcommand{\imagcomp}[1]{\text{Im}\bracks{#1}}
\newcommand{\range}[1]{\text{range}\bracks{#1}}
\newcommand{\colsp}[1]{\text{colsp}\bracks{#1}}
\newcommand{\rowsp}[1]{\text{rowsp}\bracks{#1}}
\newcommand{\tr}[1]{\text{tr}\bracks{#1}}
\newcommand{\rank}[1]{\text{rank}\bracks{#1}}
\newcommand{\proj}[2]{\text{Proj}_{#1}\bracks{#2}}
\newcommand{\projcolspX}[1]{\text{Proj}_{\colsp{\X}}\bracks{#1}}
\newcommand{\median}[1]{\text{median}\bracks{#1}}
\newcommand{\mean}[1]{\text{mean}\bracks{#1}}
\newcommand{\dime}[1]{\text{dim}\bracks{#1}}
\renewcommand{\det}[1]{\text{det}\bracks{#1}}
\newcommand{\expe}[1]{\mathbb{E}\bracks{#1}}
\newcommand{\expeabs}[1]{\expe{\abss{#1}}}
\newcommand{\expesub}[2]{\mathbb{E}_{#1}\bracks{#2}}
\newcommand{\cexpesub}[3]{\mathbb{E}_{#1}\bracks{#2~|~#3}}
\newcommand{\indic}[1]{\mathds{1}_{#1}}
\newcommand{\var}[1]{\mathbb{V}\text{ar}\bracks{#1}}
\newcommand{\mse}[1]{\mathbb{M}\text{SE}\bracks{#1}}
\newcommand{\sd}[1]{\mathbb{S}\text{D}\bracks{#1}}
\newcommand{\support}[1]{\mathbb{S}_{#1}}
\newcommand{\cov}[2]{\mathbb{C}\text{ov}\bracks{#1, #2}}
\newcommand{\corr}[2]{\mathbb{C}\text{orr}\bracks{#1, #2}}
\newcommand{\se}[1]{\text{SE}\bracks{#1}}
\newcommand{\seest}[1]{\hat{\text{SE}}\bracks{#1}}
\newcommand{\bias}[1]{\mathbb{B}\text{ias}\bracks{#1}}
\newcommand{\partialop}[2]{\dfrac{\partial}{\partial #1}\bracks{#2}}
\newcommand{\secpartialop}[2]{\dfrac{\partial^2}{\partial #1^2}\bracks{#2}}
\newcommand{\mixpartialop}[3]{\dfrac{\partial^2}{\partial #1 \partial #2}\bracks{#3}}

%shortcuts for functions
\renewcommand{\exp}[1]{\mathrm{exp}\parens{#1}}
\renewcommand{\cos}[1]{\text{cos}\parens{#1}}
\renewcommand{\sin}[1]{\text{sin}\parens{#1}}
\newcommand{\sign}[1]{\text{sign}\parens{#1}}
\newcommand{\are}[1]{\mathrm{ARE}\parens{#1}}
\newcommand{\natlog}[1]{\ln\parens{#1}}
\newcommand{\oneover}[1]{\frac{1}{#1}}
\newcommand{\overtwo}[1]{\frac{#1}{2}}
\newcommand{\overn}[1]{\frac{#1}{n}}
\newcommand{\oversqrtn}[1]{\frac{#1}{\sqrt{n}}}
\newcommand{\oneoversqrt}[1]{\oneover{\sqrt{#1}}}
\newcommand{\sqd}[1]{\parens{#1}^2}
\newcommand{\loss}[1]{\ell\parens{\theta, #1}}
\newcommand{\losstwo}[2]{\ell\parens{#1, #2}}
\newcommand{\cf}{\phi(t)}

%English language specific shortcuts
\newcommand{\ie}{\textit{i.e.} }
\newcommand{\AKA}{\textit{AKA} }
\renewcommand{\iff}{\textit{iff}}
\newcommand{\eg}{\textit{e.g.} }
\renewcommand{\st}{\textit{s.t.} }
\newcommand{\wrt}{\textit{w.r.t.} }
\newcommand{\mathst}{~~\text{\st}~~}
\newcommand{\mathand}{~~\text{and}~~}
\newcommand{\ala}{\textit{a la} }
\newcommand{\ppp}{posterior predictive p-value}
\newcommand{\dd}{dataset-to-dataset}

%shortcuts for distribution titles
\newcommand{\logistic}[2]{\mathrm{Logistic}\parens{#1,\,#2}}
\newcommand{\bernoulli}[1]{\mathrm{Bernoulli}\parens{#1}}
\newcommand{\betanot}[2]{\mathrm{Beta}\parens{#1,\,#2}}
\newcommand{\stdbetanot}{\betanot{\alpha}{\beta}}
\newcommand{\multnormnot}[3]{\mathcal{N}_{#1}\parens{#2,\,#3}}
\newcommand{\normnot}[2]{\mathcal{N}\parens{#1,\,#2}}
\newcommand{\classicnormnot}{\normnot{\mu}{\sigsq}}
\newcommand{\stdnormnot}{\normnot{0}{1}}
\newcommand{\uniform}[2]{\mathrm{U}\parens{#1,\,#2}}
\newcommand{\stduniform}{\uniform{0}{1}}
\newcommand{\exponential}[1]{\mathrm{Exp}\parens{#1}}
\newcommand{\geometric}[1]{\mathrm{Geometric}\parens{#1}}
\newcommand{\gammadist}[2]{\mathrm{Gamma}\parens{#1, #2}}
\newcommand{\negbin}[2]{\mathrm{NegBin}\parens{#1, #2}}
\newcommand{\poisson}[1]{\mathrm{Poisson}\parens{#1}}
\newcommand{\binomial}[2]{\mathrm{Binomial}\parens{#1,\,#2}}
\newcommand{\erlang}[2]{\mathrm{Erlang}\parens{#1,\,#2}}
\newcommand{\rayleigh}[1]{\mathrm{Rayleigh}\parens{#1}}
\newcommand{\multinomial}[3]{\mathrm{Multinom}_{#1}\parens{#2,\,#3}}
\newcommand{\gammanot}[2]{\mathrm{Gamma}\parens{#1,\,#2}}
\newcommand{\cauchynot}[2]{\text{Cauchy}\parens{#1,\,#2}}
\newcommand{\invchisqnot}[1]{\text{Inv}\chisq{#1}}
\newcommand{\invscaledchisqnot}[2]{\text{ScaledInv}\ncchisq{#1}{#2}}
\newcommand{\invgammanot}[2]{\text{InvGamma}\parens{#1,\,#2}}
\newcommand{\chisq}[1]{\chi^2_{#1}}
\newcommand{\ncchisq}[2]{\chi^2_{#1}\parens{#2}}
\newcommand{\ncF}[3]{F_{#1,#2}\parens{#3}}

%shortcuts for PDF's of common distributions
\newcommand{\logisticpdf}[3]{\oneover{#3}\dfrac{\exp{-\dfrac{#1 - #2}{#3}}}{\parens{1+\exp{-\dfrac{#1 - #2}{#3}}}^2}}
\newcommand{\betapdf}[3]{\dfrac{\Gamma(#2 + #3)}{\Gamma(#2)\Gamma(#3)}#1^{#2-1} (1-#1)^{#3-1}}
\newcommand{\normpdf}[3]{\frac{1}{\sqrt{2\pi#3}}\exp{-\frac{1}{2#3}(#1 - #2)^2}}
\newcommand{\normpdfvarone}[2]{\dfrac{1}{\sqrt{2\pi}}e^{-\half(#1 - #2)^2}}
\newcommand{\chisqpdf}[2]{\dfrac{1}{2^{#2/2}\Gamma(#2/2)}\; {#1}^{#2/2-1} e^{-#1/2}}
\newcommand{\invchisqpdf}[2]{\dfrac{2^{-\overtwo{#1}}}{\Gamma(#2/2)}\,{#1}^{-\overtwo{#2}-1}  e^{-\oneover{2 #1}}}
\newcommand{\uniformdiscrete}[1]{\mathrm{Uniform}\parens{\braces{#1}}}
\newcommand{\exponentialpdf}[2]{#2\exp{-#2#1}}
\newcommand{\poissonpdf}[2]{\dfrac{e^{-#1} #1^{#2}}{#2!}}
\newcommand{\binomialpdf}[3]{\binom{#2}{#1}#3^{#1}(1-#3)^{#2-#1}}
\newcommand{\rayleighpdf}[2]{\dfrac{#1}{#2^2}\exp{-\dfrac{#1^2}{2 #2^2}}}
\newcommand{\gammapdf}[3]{\dfrac{#3^#2}{\Gamma\parens{#2}}#1^{#2-1}\exp{-#3 #1}}
\newcommand{\cauchypdf}[3]{\oneover{\pi} \dfrac{#3}{\parens{#1-#2}^2 + #3^2}}
\newcommand{\Gammaf}[1]{\Gamma\parens{#1}}

%shortcuts for miscellaneous typesetting conveniences
\newcommand{\notesref}[1]{\marginpar{\color{gray}\tt #1\color{black}}}

%%%% DOMAIN-SPECIFIC SHORTCUTS

%Real analysis related shortcuts
\newcommand{\zeroonecl}{\bracks{0,1}}
\newcommand{\forallepsgrzero}{\forall \epsilon > 0~~}
\newcommand{\lessthaneps}{< \epsilon}
\newcommand{\fraccomp}[1]{\text{frac}\bracks{#1}}

%Bayesian related shortcuts
\newcommand{\yrep}{y^{\text{rep}}}
\newcommand{\yrepisq}{(\yrep_i)^2}
\newcommand{\yrepvec}{\bv{y}^{\text{rep}}}


%Probability shortcuts
\newcommand{\SigField}{\mathcal{F}}
\newcommand{\ProbMap}{\mathcal{P}}
\newcommand{\probtrinity}{\parens{\Omega, \SigField, \ProbMap}}
\newcommand{\convp}{~{\buildrel p \over \rightarrow}~}
\newcommand{\convLp}[1]{~{\buildrel \Lp{#1} \over \rightarrow}~}
\newcommand{\nconvp}{~{\buildrel p \over \nrightarrow}~}
\newcommand{\convae}{~{\buildrel a.e. \over \longrightarrow}~}
\newcommand{\convau}{~{\buildrel a.u. \over \longrightarrow}~}
\newcommand{\nconvau}{~{\buildrel a.u. \over \nrightarrow}~}
\newcommand{\nconvae}{~{\buildrel a.e. \over \nrightarrow}~}
\newcommand{\convd}{~{\buildrel d \over \rightarrow}~}
\newcommand{\nconvd}{~{\buildrel d \over \nrightarrow}~}
\newcommand{\withprob}{~~\text{w.p.}~~}
\newcommand{\io}{~~\text{i.o.}}

\newcommand{\Acl}{\bar{A}}
\newcommand{\ENcl}{\bar{E}_N}
\newcommand{\diam}[1]{\text{diam}\parens{#1}}

\newcommand{\taua}{\tau_a}

\newcommand{\myint}[4]{\int_{#2}^{#3} #4 \,\text{d}#1}
\newcommand{\laplacet}[1]{\mathscr{L}\bracks{#1}}
\newcommand{\laplaceinvt}[1]{\mathscr{L}^{-1}\bracks{#1}}
\renewcommand{\max}[1]{\text{max}\braces{#1}}
\renewcommand{\min}[1]{\text{min}\braces{#1}}

\newcommand{\Vbar}[1]{\bar{V}\parens{#1}}
\newcommand{\expnegrtau}{\exp{-r\tau}}
\newcommand{\cprob}[2]{\prob{#1~|~#2}}
\newcommand{\ck}[2]{k\parens{#1~|~#2}}

%%% problem typesetting
\newcommand{\problem}{\vspace{0.2cm} \noindent {\large{\textsf{Problem \arabic{probnum}~}}} \addtocounter{probnum}{1}}
%\newcommand{\easyproblem}{\ingreen{\noindent \textsf{Problem \arabic{probnum}~}} \addtocounter{probnum}{1}}
%\newcommand{\intermediateproblem}{\noindent \inyellow{\textsf{Problem \arabic{probnum}~}} \addtocounter{probnum}{1}}
%\newcommand{\hardproblem}{\inred{\noindent \textsf{Problem \arabic{probnum}~}} \addtocounter{probnum}{1}}
%\newcommand{\extracreditproblem}{\noindent \inpurple{\textsf{Problem \arabic{probnum}~}} \addtocounter{probnum}{1}}

\newcommand{\easysubproblem}{\ingreen{\item}}
\newcommand{\intermediatesubproblem}{\inyellow{\item}}
\newcommand{\hardsubproblem}{\inred{\item}}
\newcommand{\extracreditsubproblem}{\inpurple{\item}}


\newcounter{numpts}
\setcounter{numpts}{0}


%\newcommand{\subquestionwithpoints}[1]{\addtocounter{numpts}{#1} \item \ingray{[#1 pt]}~~} %  / \arabic{numpts} pts
\newcommand{\subquestionwithpoints}[1]{\addtocounter{numpts}{#1} \item \ingray{[#1 pt / \arabic{numpts} pts]}~~}  
\newcommand{\truefalsesubquestionwithpoints}[1]{\subquestionwithpoints{#1} Record the letter(s) of all the following that are \textbf{true} in general. At least one will be true.}
\newcommand{\multchoicewithpoints}[2]{\subquestionwithpoints{#1} #2}

\newcounter{nummin}
\setcounter{nummin}{0}

\usepackage{accents}
\newlength{\dhatheight}
\newcommand{\doublehat}[1]{%
    \settoheight{\dhatheight}{\ensuremath{\hat{#1}}}%
    \addtolength{\dhatheight}{-0.35ex}%
    \hat{\vphantom{\rule{1pt}{\dhatheight}}%
    \smash{\hat{#1}}}}
\newcommand{\thetahathat}{\doublehat{\theta}}

%\newcommand{\subquestionwithpoints}[1]{\addtocounter{numpts}{#1} \item \ingray{[#1 pt]}~~} %  / \arabic{numpts} pts
\newcommand{\timedsection}[1]{\addtocounter{nummin}{#1}{[#1min] \ingray{(and \arabic{nummin}min will have elapsed)}}}  
%\newcommand{\timedsection}[1]{\addtocounter{nummin}{#1}{[#1 min]}}



%%%unchanged
\newcommand{\coursewebpageurl}{https://github.com/kapelner/QC_\coursedept_\coursenumber_\semester_\the\year}
\newcommand{\coursewebpagelink}{\href{\coursewebpageurl}{course homepage}}
\newcommand{\discordurl}{https://discord.com/channels/1324190933906096180}
\newcommand{\discordlink}{\href{\discordurl}{discord}}

%%edit these!!
\newcommand{\coursedept}{Math}
\newcommand{\coursenumber}{343}
\newcommand{\sixhundredsection}{643}
\newcommand{\coursenumbercrosslisted}{/ 643}
\newcommand{\professorname}{Adam Kapelner, PhD.}
\newcommand{\professorcontactinfo}{@kapelner on \discordlink~in the relevant channel}
\newcommand{\professoroffice}{604 Kiely Hall}
\newcommand{\semester}{Spring}
\newcommand{\numcredits}{3}
\newcommand{\lectimeandloc}{Tues and Thurs 5:20 -- 6:35PM / KY 258}
\newcommand{\requiredlabtimeandloc}{}
\newcommand{\tataofficehourtimeandloc}{see \coursewebpagelink}
\newcommand{\numtheoryhws}{3}
\newcommand{\lastdatetimetohandinhomeworks}{Dec 15 at noon}
\newcommand{\midtermonedateandlocation}{Thursday, October 7 on zoom during class time}
\newcommand{\midtermtwodateandlocation}{Thursday, November 11 on zoom during class time}
\newcommand{\finaldateandlocation}{TBD but on zoom}


\input{../../syllabi/_header}



\section*{Course Overview}

MATH 343 / 643. Computational Statistics for Data Science. 3 hr.; 3 cr. Prereq.: MATH 341 or 641. Coreq.: MATH 342W or 642. Mixture models, EM algorithm, Metropolis-within-Gibbs sampling, permutation tests, the bootstrap, the Kaplan-Meier estimator, the Cox model, T and F tests for the linear model, Gauss-Markov theorem, Bayesian linear regression: Ridge and Lasso. Causality and the randomized experiment, randomization tests. Focus on computation. Special topics. Students cannot receive credit for both: MATH 343 and 643. Fall, Spring \\

\section*{The Four Data Science Core Classes}

This course is the last and most advanced of the four data science core classes as it builds on the previous three. In Math 340 you learn probability tools and a repertoire of random variable models. In Math 341, you learned the foundations of statistics from both the Frequentist and Bayesian perspectives. Here, we continue with more advanced Frequentist and Bayesian methods. The class will be mostly computational as you now are learning / learned computational skills in Math 342W. Thus, \textbf{this is not your typical mathematics course.}  This course will do lots of modeling of real-world situations using data via the \texttt{R} statistical language.

We will also be able to delve more deeply into the concepts of Math 342W including linear regression from a statistical point of view deriving all the usual linear regression techniques you may have seen in Econometrics. Further, we will also delve into the abyss of modeling and ask questions about correlation vs causation and develop causal models from the ground up. We will discuss the theory of randomization, clinical trials and A/B testing.

If we have time, we will get to advanced machine learning methods such as clustering, reinforcement learning and deep learning.

\input{../../syllabi/_DSS_core}

Examining the above two flowcharts, we note that MATH 341 and 343 form a series of two statistics courses: the first, theoretical with traditional topics and the second, computational with modern topics. Both heavily rely on the theoretical topics taught in MATH 340.

%\pagebreak
\section*{Tentative Day-by-Day Schedule}

Lectures and their topics with rough time estimates per topic are below:

\begin{enumerate}
\item[D1, Lec 1] [15min] review of Gibbs sampling;  [20min] background on Markov chains and invariant distributions; [20min] sketch of proof of why Gibbs sampling converges; [25min] burning and thinning with discussion of autocorrelation

\item[D2, Lec 2] [25min] Poisson count model with hurdle; [25min] the \texttt{stan / rstan} package for model-fitting and inference; [25min] Poisson change point model

\item[D3, Lec 3] [50min] Metropolis-Hastings Algorithm with application to Poisson Regression in \texttt{stan}; [25min] Permutation testing for different DGP's among two populations

\item[D4, Lec 4] [25min] Choice of test statistics for permutation testing for different DGP's among two populations; [50min] Bootstrap for confidence interval construction and testing with examples

\item[D5, Lec 5] [50min] Estimation for mixture model with Expectation-Maximization algorithm (EM); [25min] Gibbs sampler for mixture model inference

\item[D6, Lec 6] [10min] introduction to survival analysis / churn modeling; [20min] review of the Weibull rv; [10min] invariance of MLE thm; [15min] right censoring at a fixed time point; [20min] Weibull likelihood with right-censoring

%put invariance of MLE into 341!!!
%change definition of c_i vector

\item[D7, Lec 7] [15min] empirical survival function as the sum of indicators; [15min] empirical survival function as the product limit estimator; [15min] inference for survival; [15min] product limit estimator with duplicate survivals; [15min] empirical survival function with right censoring at a fixed time point

\item[D8, Lec 8] [15min] Right censoring at arbitrary times; [30min] Kaplan-Meier survival function estimator; [10min] inference for one-sample Kaplan-Meier survival estimator; [20min]  inference for two-sample Kaplan-Meier survival estimator with log-rank test

\item[D9] \inblue{Midterm I Review}
\item[D10] \inred{Midterm I}

\item[D11, Lec 9] [10min] review of linear model candidate set and optimal element; [20min] motivation of linear model error multivariate normal core assumption; [10min] proof of normality of OLS estimators; [25min] Cochran's theorem applied to error terms; [10min] computing the correct denominator in the unbiased estimator of $\sigma^2$

\item[D12, Lec 10] [10min] Proof of the normality of OLS predictions and OLS residuals; [15min] proof that residuals and slope estimators are independent; [30min] proof that the test statistic for the OLS estimators are Student's t-distributed; [20min] proof that the test statistic for the expectation of a response is Student's t-distributed

\item[D13, Lec 11] [15min] definition and motivation of adjusted $R^2$; [20min] proof that the test statistic for the the response is Student's t-distributed, prediction intervals; [40min] proving the omnibus $F$ test

\item[D14, Lec 12] [10min] proving that $R^2$ is beta-distribution under the null of no linear signal in the covariates; [65min] proving the partial $F$ test for arbitrarily nested (full / reduced) models

\item[D15, Lec 13] [15min] proof that the OLS estimator is the maximum likelihood estimator (MLE); [60min] proof that the ridge estimator is the maximum a posteriori estimator under iid normal prior on the slope parameters, ridge regression demos, proof that ridge regression is biased

\item[D16, Lec 14] [75min] proof that the lasso estimator is the maximum a posteriori estimator under iid laplace prior on the slope parameters, lasso vs ridge demos, lasso as a \qu{variable selection} preprocessing algorithm

\item[D17, Lec 15] [10min] introduction to robust linear regression inference without the core error assumption; [10min] employing the bootstrap for linear regression inference; [25min] linear regression inference for mean centered homoskedastic errors with asymptotically normal-distributed estimators and Wald testing; [30min] linear regression inference for mean centered heteroskedastic errors with asymptotically normal-distributed Huber-White estimators and Wald testing

\item[D18, Lec 16] [15min] inference for single effects in probability regression models, a type of generalized linear model (GLM) using MLE's and Wald tests; [10min] likelihood ratio test (LRT) with arbitrarily nested (full / reduced) models; [10min] LRT for probability regression models; [15min] Poisson regression, the log-linear mean link function, and inference for single effects with Wald testing and inference for multiple effects with the LRT; [15min] Negative binomial regression, reparameterization to obtain the log-linear mean link function, and inference for single effects with Wald testing and inference for multiple effects with the LRT, discussion of nuisance parameter; [10min] review of Weibull regression with censoring, the log-linear mean link function, inference for single effects with Wald testing and inference for multiple effects with the LRT

\item[D19, Lec 17] [45min] definition of hazart rates / hazard functions, expressing the PDF with the hazard rate and an integral representation of the survival function; [30min] the Cox proportional hazards model (cox PH), the likelihood of the survival DGP in cox PH modeling

\item[D20] \inblue{Midterm II Review}
\item[D21] \inred{Midterm II}

\item[D22, Lec 18] [35min] expression that allows inference for the cox PH model, demos of cox PH modeling vs Weibull modeling; [15min] spurious correlations; [15min] causal diagrams as directed acyclic graphs (DAG's); [10min] definition of causality as manipulations within the DAG



%[20min] causal inference vs statistical inference, counfounding variable, an illustration of one scenario with confounding and one scenario without confounding; [15min] definition of a two-arm treatment vs control experiment, definition of assignment / allocation / manipulation; [10min] the Rubin causal model, counterfactuals, additive treatment effect, selection bias inducing bias in the naive treatment estimator; [30min] response model linear in confounder and noise, noise as an approximate normal realization, the explicit bias term and its explanation;

%[30min] randomized experiments, bernoulli trial design, completely randomized trial design, statment that causal estimates are unbiased over errors and randomized assignments, statement that bias is small in any random assignment; [45min] the randomization test vs. the permutation test, hypothesis testing and confidence intervals

\item[D23, Lec 19] [20min] Three common DAGs; [10min] spurious correlation vs correlation vs causation, correlation does not imply causation, but implies causation \emph{somewhere}; [10min] the full picture of y, z's, x's from 342; [45min] proper interpretation of coefficients in G/LM's

\item[D24, Lec 20] [15min] more proper interpretation of coefficients in G/LM's; [10min] the impossibility of assessing causation through a scatterplot or regression; [20min] confounding and regressions with and without confounders for the three common DAGs; [15min] proper interpretation of causal coefficients in G/LM's as \emph{manipulations}; [15min] Simpson's Paradox


%70min
\item[D25, Lec 21] [10min] Berkson's Paradox; [10min] Fully path-blocked and partially path-blocked DAGs; [10min] Decision tree for all regression scenarios; [45min] introduction to treatment-control experiments (A/B testing)

%[55min] Derivation of T-test and F-test and chi-squared test linear regression estimator; 


%[20min] derivation of variance-covariance matrix of the least squared estimators

%75min
\item[D26, Lec 22] [15min] Experimental allocation vs experimental design; [25min] unbiasedness of the treatment coefficient for the population average treatment effect in the linear model; [25min] optimal design for the linear model with additive treatment effect; [10min] Finite experimental bias

%[30min] Introduction to neural networks; [45min] estimation of parameters in neural networks using optimization algorithms

%80min
\item[D27, Lec 23] [10min] Completely randomized design, balanced completely randomized design; [15min] Fisher's randomization test; [10min] Restricted randomization; [10min] Blocking design; [10min] Pairwise-matching design; [10min] Rerandomization design; [10min] multilogit/probit regression introduction

%[30min] Introduction to deepl learning; [45min] convolutional neural networks for image modeling

\item[D28] \inblue{Final Review}

\end{enumerate}





\subsection*{Prerequisites}

MATH 341 / 641 or a foundations of Frequentist and Bayesian statistics course. Critical is coverage of hypothesis testing, confidence intervals, credible regions, maximum likelihood, priors and posteriors, maximum a posteriori estimation. Implicitly, MATH 340 / 640 is also a prerequisite. At times, we will be drawing on these concepts or extending these concepts (e.g. hazard rates for survival random variables) so keep those notes handy.

\subsection*{Corequisites}

MATH 342W / 642 or a foundations of data science course. Critical is coverage of supervised learning, prediction concepts, basic computing using the \texttt{R} language, OLS linear regression with matrix algebra and logistic regression.

\section*{Course Materials}

\paragraph{Textbook:} I will be referencing Larry Wasserman's \emph{All of Statistics: A concise course in statistical inference} which can be purchased on \href{https://www.amazon.com/dp/0387402721}{Amazon} and Casella and Berger's \emph{Statistical Inference} which can be purchased on \href{https://www.amazon.com/dp/8131503941}{Amazon}. There is no excuse not to have these books. They are \textit{required}. However, I will not ususally be teaching \qu{from the book} --- most of the material in the class comes from the lecture notes. The textbooks are a way to get ``another take'' on the material and they will only cover about only half of the material done in class. For the other half, you will have to make use of other resources. I also recommend Rice's \emph{Mathematical Statistics and Data Analysis}, 3rd edition which can be purchased on \href{https://www.amazon.com/dp/0534399428}{Amazon} as well but I will not reference it during class.

\paragraph{Computer Software:} During lectures, there will be demos using \texttt{R} which is a free, open source statistical programming language and console. You can download it from: \url{http://cran.mirrors.hoobly.com/}.As this course is coreq'd with MATH 342W, this course has a lot of programming in \texttt{R} for the homeworks.

\input{../../syllabi/_calculator}

%\input{../../syllabi/_the650section}

\input{../../syllabi/_use_of_discord}

\input{../../syllabi/_announcements_on_discord}

\input{../../syllabi/_standard_class_meetings}

%\input{../../syllabi/_jewish_holiday_reschedule}

%\input{../../syllabi/_zoom_policies}

%\input{../../syllabi/_lecture_upload}

\section*{Homework}

There will be \numtheoryhws~theory homework assignments and \numtheoryhws~practice homework assignments (labs). Homeworks will be assigned and placed on the \coursewebpagelink~ and will usually be due a week later in class. Homework will be \textbf{graded} out of 100 with extra credit getting scores possibly $> 100$. I will be doing the grading and will grade an \textit{arbitrary subset of the assignment} which is determined after the homework is handed in. 

\input{../../syllabi/_theory_hws_submission_text}
\input{../../syllabi/_philosophy_hws}

\input{../../syllabi/_time_spent_hws_3_cr}

\input{../../syllabi/_late_hw_policy}

\input{../../syllabi/_latex_hw_bonus_policy}

\input{../../syllabi/_hw_ec_policy}

\section*{Examinations}

\input{../../syllabi/_examination_text}

\input{../../syllabi/_standard_exam_schedule}

\subsection*{Exam Policies and Materials}

\input{../../syllabi/_examination_policies}

%\input{../../syllabi/_zoom_examination_policies}

\input{../../syllabi/_standard_cheat_sheet_policy}


\input{../../syllabi/_cheating_on_exams_and_missing_exams}
\input{../../syllabi/_special_services}

\input{../../syllabi/_class_participation}

%\input{../../syllabi/_zoom_attendance}

\input{../../syllabi/_343_grading_and_grading_policy}

\subsection*{The Grade Distribution}

As this is a small and advanced class, the class curve will be quite generous. I'm expecting approximately 40\% A's and 40\% B's. If you do your homework and demonstrate understanding on the exams, you should expect to be rewarded with an A or a B. C's are for those who \qu{dropped out} somewhere mid-semester or who cano only demonstrate rudimentary understanding. F's are rare and are for those who miss exams or are not demonstrating any understanding.

\input{../../syllabi/_grade_checking_on_gradesly}

\input{../../syllabi/_auditing_policy}

\end{document}